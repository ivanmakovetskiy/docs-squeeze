\chapter{Cоздание проекта в среде MS Visual Studio с поддержкой OpenMP}

\begin{enumerate}
   \item Создадим на рабочем столе папку «Маковецкий».

    \item Запустим Microsoft Visual Studio 2010.

    \item Создадим проект. Для этого выберем пункт в меню File -> New -> Project, или нажмем Ctrl+Shift+N.

    \item В окне New Project в раскрывающемся списке Visual C++ выберем Win32. В подокне в середине выберем Win32 Console Application. Внизу введем имя проекта Name и место расположения проекта Location и нажмем кнопку OK.
    \item В открывшемся окне Win32 Application Wizard --- example1 нажмем кнопку Next, и затем в Additional options поставим галочку напротив Empty project. Нажмем кнопку Finish.
        \item Теперь создадим файл с кодом приложения. Выберите пункт в ме-
        ню Project
        -> Add New Item, или нажмите Ctrl+Shift+A. В категории Visual
        C++ выберем подкатегорию Code. В подокне в середине установим С++
        File (.cpp). Введите имя файла, например, source, и нажмите кнопку Add.
        \item В открывшемся окне source.cpp введем следующий код на языке С:
            \begin{lstlisting}
int main() {
    return 0;
}
            \end{lstlisting}
            Сохраним файл, выбрав пункт меню File -> Save source.cpp, или нажав Ctrl+S.
        \item Для компиляции приложения выберем пункт меню Debug -> Build Solution, или нажмем F7.
        \item Для запуска приложения выберем пункт меню Debug -> Start Without Debugging, или нажмем Ctrl+F5.
        \item Для включения поддержки OpenMP установим дополнительные параметры компиляции проекта:
            \begin{itemize}
            \item В главном меню выберем Project-> Имя проекта Properties.
            \item В открывшемся окне выберем Configuration Properties / C/C++ /
                Language. Установим для опции OpenMP Support значение Yes
                (/openmp).
            \end{itemize}
        \item Для компиляции приложения нажмем F7.
        \item Для запуска приложения нажмем Ctrl+F5.
\end{enumerate}

\FloatBarrier
